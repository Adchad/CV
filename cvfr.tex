
\documentclass[a4paper,20pt]{article}

\usepackage{latexsym}
\usepackage[empty]{fullpage}
\usepackage{titlesec}
\usepackage{marvosym}
\usepackage[usenames,dvipsnames]{color}
\usepackage{verbatim}
\usepackage{enumitem}
\usepackage[pdftex]{hyperref}
\usepackage{fancyhdr}

\pagestyle{fancy}
\fancyhf{} % clear all header and footer fields
\fancyfoot{}
\renewcommand{\headrulewidth}{0pt}
\renewcommand{\footrulewidth}{0pt}

% Adjust margins
\addtolength{\oddsidemargin}{-0.530in}
\addtolength{\evensidemargin}{-0.375in}
\addtolength{\textwidth}{1in}
\addtolength{\topmargin}{-.45in}
\addtolength{\textheight}{1in}

\urlstyle{rm}

\raggedbottom
\raggedright
\setlength{\tabcolsep}{0in}
% Sections formatting
\titleformat{\section}{
  \vspace{-10pt}\scshape\raggedright\large
}{}{0em}{}[\color{black}\titlerule \vspace{-6pt}]

%-------------------------
% Custom commands
\newcommand{\resumeItem}[2]{
  \item\small{
    \textbf{#1}{: #2 \vspace{-2pt}}
  }
}

\newcommand{\resumeItemWithoutTitle}[1]{
  \item\small{
    {\vspace{-2pt}#1}
  }
}

\newcommand{\resumeSubheading}[4]{
  \vspace{-1pt}\item
    \begin{tabular*}{0.97\textwidth}{l@{\extracolsep{\fill}}r}
      \textbf{#1} & #2 \\
      \textit{#3} & \textit{#4} \\
    \end{tabular*}\vspace{-5pt}
}



\newcommand{\resumeSubItem}[2]{\resumeItem{#1}{#2}\vspace{-3pt}}
\newcommand{\resumeSubItemWithoutTitle}[1]{\resumeItemWithoutTitle{#1}\vspace{-3pt}}
\renewcommand{\labelitemii}{$\circ$}

\newcommand{\resumeSubHeadingListStart}{\begin{itemize}[leftmargin=*]}
\newcommand{\resumeSubHeadingListEnd}{\end{itemize}}
\newcommand{\resumeItemListStart}{\begin{itemize}}
\newcommand{\resumeItemListEnd}{\end{itemize}\vspace{-5pt}}



\begin{document}

\begin{tabular*}{\textwidth}{l@{\extracolsep{\fill}}r}
		\textbf{{\LARGE Adam Chader}} & Email : \href{mailto:adam.chader@telecom-paris.fr}{adam.chader@telecom-paris.fr}\\
		\href{http://github.com/Adchad}{Github : \textit{https://github.com/Adchad}} & Tel : 0637478629 \\
		\href{http://linkedin.com/in/adam-chader}{Linkedin : \textit{http://linkedin.com/in/adam-chader}} \\
\end{tabular*}




\vspace{5pt}

\section{Scolarité}
 \resumeSubHeadingListStart
  \resumeSubheading
  {Institut Polytechnique de Paris}{Saclay, France}
  {Master de Recherche}{sept 2021 - sept 2022}
   \resumeItemListStart 
    \resumeItem
    {Systèmes Parallèles et Distribués}{Calcul Haute Performance, Infrastructures du Cloud, Middlewares, Architectures des Logiciels Répartis}
   \resumeItemListEnd 
  \resumeSubheading
  {Télécom Paris}{Saclay, France}
  {Diplôme d'ingénieur}{sept 2019 - sept 2022}
   \resumeItemListStart 
    \resumeItem
    {Systèmes Distribués}{Architecture logicielle et gestion de projet, Informatique distribuée, Algorithmes distribués, Blockchain}
    \resumeItem
    {Data Science}{Regression linéaire, Optimization pour le Machine Learning, Statistiques avancées, Programmation logique (Prolog), Data Mining}
   \resumeItemListEnd 
  \resumeSubheading
  {Lycée Chaptal}{Paris, France}
  {Classes préparatoires: PCSI/PSI*}{sept 2017 - juillet 2019}
  \resumeSubheading
  {Lycée St-Michel-des-Batignolles}{Paris, France}
  {Bac S}{juillet 2017}
 \resumeSubHeadingListEnd

\vspace{5pt}

\section{Expériences | Projets}
 \resumeSubHeadingListStart
  \resumeSubheading
  {Degradable Data Structures | Projet de Recherche}{ }
  {Institut Polytechnique de Paris}{sept 2021 - sept 2022}
   \resumeItemListStart 
    \resumeItemWithoutTitle
    {Améliorations des performances d'applications distribuées en enlevant les bottlenecks liés aux accès mémores concurrents. Cela est rendu possible par la dégradation des structures de données distribuées. Etude de bases de données distribuées (Apache Ignite), et modification du code source. }
   \resumeItemListEnd 
  \resumeSubheading
  {Hadoop MapReduce | SLR207}{\textit{\href{https://github.com/Adchad/SLR207}{https://github.com/Adchad/SLR207}}}
  {Télécom Paris}{mars 2020 - juin 2020 }
   \resumeItemListStart 
    \resumeItemWithoutTitle
    {Réimplémentation de l'algorithme MapReduce de Hadoop \textit{from scratch} en utilisant ssh. L'objectif de ce projet était d'apprendre l'algorithme MapReduce, qui est très classique dans les systèmes distribués. L'implémentation est en Java. On étudie ensuite les performances de l'algorithme et on essaye de prouver empiriquement la loi d'Amdahl.}
   \resumeItemListEnd 
  \resumeSubheading
  {Paxos | SLR210}{\textit{\href{https://github.com/Adchad/slr210}{https://github.com/Adchad/slr210}}}
  {Télécom Paris}{mars 2020 - juin 2020 }
   \resumeItemListStart 
    \resumeItemWithoutTitle
    {Une implémentation \textit{from scratch} de l'algorithme Paxos/Synod en utilisant le systèmes d'acteurs Akka. Le but de cet algorithme est de réaliser une Réplication de Machine à état, en résolvant une version moins lourde du consensus, appelée \textit{Obstruction-Free Consensus}.}
   \resumeItemListEnd 
  \resumeSubheading
  {PresSync}{\textit{\href{https://github.com/Adchad/PresSync}{https://github.com/Adchad/PresSync}}}
  {Télécom Paris}{juin 2020}
   \resumeItemListStart 
    \resumeItemWithoutTitle
    {Gestionnaire de présentations utilisant Reveal.Js. L'objectif de ce projet était de développer une solution permettant de partager une présentation entre un élève et un professeur, et de les synchroniser. Le système utilise un backend en Python avec Flask et SocketIO, et un frontend Javascript classique.}
   \resumeItemListEnd 
  \resumeSubheading
  {CleanLake}{ }
  {Télécom Paris}{sept 2019 - juin 2020}
   \resumeItemListStart 
    \resumeItemWithoutTitle
    {Bateau robot pour nettoyer les océans. Le projet a été réalisé pour PACT à Télécom Paris. Le bateau utilise une carte embarquée pour le pilotage et la détection du plastique sur les surfaces d'eau. La carte est programmée en C++, et le projet est integré en Java.}
   \resumeItemListEnd 
 \resumeSubHeadingListEnd


\vspace{5pt}

\section{Compétences}
 \resumeSubHeadingListStart
  \resumeSubItem{Languages de programmation}{~Java, C, C++, Python, JavaScript, Bash, SQL, Prolog}
  \resumeSubItem{Outils}{Git, Linux/Unix, \LaTeX}
  \resumeSubItem{Cloud}{~Kubernetes, Docker, GCP}
  \resumeSubItem{Calcul Haute Performance}{~CUDA, MPI, OpenMP}
  \resumeSubItem{Data Science}{Keras/TensorFlow, ScikitLearn}
  \resumeSubItem{Langues}{~Français(Langue Maternelle), Anglais(Bilingue), Allemand(Scolaire)}
 \resumeSubHeadingListEnd

\vspace{5pt}

\section{Divers}
 \resumeSubHeadingListStart
  \resumeSubItem{Jobs étudiants}{~Professeur particulier à Complétude}
  \resumeSubItem{Sports}{Natation Haut Niveau, Basketball}
  \resumeSubItem{Musique}{Violoncelle et Basse au conservatoire}
\resumeSubHeadingListEnd

\end{document}
